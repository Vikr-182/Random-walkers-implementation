%%% Template originaly created by Karol Kozioł (mail@karol-koziol.net) and modified for ShareLaTeX use

\documentclass[a4paper,11pt]{article}

\usepackage[T1]{fontenc}
\usepackage[utf8]{inputenc}
\usepackage{graphicx}
\usepackage{xcolor}
\usepackage{amsfonts}%For Maths fonts.
\usepackage{amssymb}%For math symbols.
\usepackage{amsmath}%For the math.
\usepackage{amsthm}
\usepackage{amsfonts}
\usepackage[utf8]{inputenc}%For the theorems,proofs and corollaries.
\newtheorem*{remark}{Remark}%For including remarks here and there.
\usepackage[english]{babel}
\usepackage{fullpage}
\usepackage{fancyvrb}
\usepackage{algorithm}
\usepackage{graphicx}%For the image.
\graphicspath{ ./} 
\usepackage{tikz}
\usepackage{xcolor}
\usepackage{float}
\usepackage{verbatim}
\usepackage{bera}% optional: just to have a nice mono-spaced font
\usepackage{listings}
\usepackage{xcolor}
\usepackage{pythontex}
\colorlet{punct}{red!60!black}
\definecolor{background}{HTML}{EEEEEE}
\definecolor{delim}{RGB}{20,105,176}
\colorlet{numb}{magenta!60!black}
\usetikzlibrary{shapes.geometric, arrows}
\theoremstyle{definition}
\newtheorem{definition}{Definition}
\theoremstyle{plain}
\usepackage{xcolor}
\usepackage{dirtree}
\definecolor{light-gray}{gray}{0.95}
\definecolor{light}{gray}{0.10}
\newcommand{\code}[1]{\colorbox{light-gray}{\texttt{#1}}}
\parindent0in
\pagestyle{plain}
\thispagestyle{plain}
\newtheorem{solution}{Solution}
\newcommand{\assignment}{\textbf{Assignment-4 Report}}
\newcommand{\duedate}{October 19,2019}

\usepackage{tgtermes}

\usepackage[
pdftitle={SCIENCE - 1 ASSIGNMENT} , 
pdfauthor={Joe Doe, Some University},
colorlinks=true,linkcolor=blue,urlcolor=blue,citecolor=blue,bookmarks=true,
bookmarksopenlevel=2]{hyperref}
\usepackage{multicol}
\usepackage{tikz}

\usepackage{geometry}
\geometry{total={210mm,297mm},
	left=25mm,right=25mm,%
	bindingoffset=0mm, top=20mm,bottom=20mm}


\linespread{1.3}

\newcommand{\linia}{\rule{\linewidth}{0.5pt}}

% custom theorems if needed
\newtheoremstyle{mytheor}
{1ex}{1ex}{\normalfont}{0pt}{\scshape}{.}{1ex}
{{\thmname{#1 }}{\thmnumber{#2}}{\thmnote{ (#3)}}}

\theoremstyle{mytheor}
\newtheorem{defi}{Definition}

% my own titles
\makeatletter
\renewcommand{\maketitle}{
	\begin{center}
		\vspace{2ex}
		{\huge \textsc{\@title}}
		\vspace{1ex}
		\\
		\linia\\
		\@author \hfill \@date
		\vspace{4ex}
	\end{center}
}
\makeatother
%%%

% custom footers and headers
\usepackage{fancyhdr,lastpage}
\pagestyle{fancy}
\lhead{}
\chead{}
\rhead{}
\lfoot{Assignment \textnumero{} 5}
\cfoot{}
\rfoot{Page \thepage\ /\ \pageref*{LastPage}}
\renewcommand{\headrulewidth}{0pt}
\renewcommand{\footrulewidth}{0pt}
%

%%%----------%%%----------%%%----------%%%----------%%%

\begin{document}
	
	\title{SCIENCE-$1$ ASSIGNMENT}
	
	\author{Vikrant Dewangan}
	
	\date{12/10/2019}
	
	\maketitle
	\section*{Problem Statement}
	Two drunk start out together at the origin, each having equal probability of making a step to the left or right along the $x$-axis. Find the probability that they meet again after $N$ steps.\\
	\textbf{\underline{Solution:}}
		\\ Let us assume $p_i(x,N)$ denote the probability of random walker $i$ being at $x$ = $x$ after $N$ steps.
		Let $P(N)$ denotes probabilty of meeting after N steps.
		We have, from total probability theorem,
			\begin{eqnarray}
			P(N) = \sum_{x = -N}^{x = +N} p_1(x,N) \cdot p_2(x,N)
 			\end{eqnarray}
			as $x$ can only take values from $-N$ to $+N$.
		\\
		We have for a given $N$, let $L_i$ be the steps taken by $i^{th}$ towards the left and $R_i$ be the steps taken towards the right.We have,
		\begin{align*}
		L_i + R_i &= N\\
		R_i - L_i &= x
		\end{align*}
		\begin{flushright}
			for $i \in \{1,2\}$	
		\end{flushright}
	Thus we get 
	\begin{align*}
		L_i &= \frac{N-x}{2};\\
		R_i &= \frac{N+x}{2};
	\end{align*}
		Thus we get $p_i(x,N)$ is the probability of taking $L_i$ steps to the left (or $R_i$ steps to the right ).
		\begin{align}
			p_i(x,N) &= \frac{{N\choose \frac{N+x}{2}}}{2^N}
		\end{align}
		Thus we get from (1) and (2),
		\begin{align*}
		P(N) &= \sum_{x=-N}^{x=+N}\frac{{N\choose \frac{N+x}{2}}}{2^N} \cdot \frac{{N\choose \frac{N+x}{2}}}{2^N}\\
		&= \sum_{x=0}^{x=+N}\left(\frac{{N\choose x}}{2^N}\right)^2
		\end{align*}	
		Consider the binomial expansion - 
		\begin{align}
		{N\choose 0}^2 + {N\choose 1}^2 + {N\choose 2}^2 + \ldots + {N\choose N}^2
		\end{align}
		which we know, is equal to 
		\begin{align}
		{2N\choose N}
		\end{align}
		Thus 
		\begin{eqnarray}
		P(N) = \frac{{2N\choose N}}{4^N}
		\end{eqnarray}
	
\end{document}.
